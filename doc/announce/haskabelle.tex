\documentclass{article}

\usepackage{hcar}

\begin{document}

\begin{hcarentry}{Haskabelle}
\report{Florian Haftmann}
\status{working}
%\participants{(PARTICIPANTS OTHER THAN MYSELF)}% optional
\makeheader

Since Haskell is a pure language, reasoning about equational semantics of Haskell programs is conceptually simple.  To facilitate machine-aided verification of Haskell programs further, we have developed a converter from Haskell source files to Isabelle theory files: Haskabelle.

Isabelle itself is a generic proof assistant.  It allows mathematical formulas to be expressed in a formal language and provides tools for proving those formulas in a logical calculus.  One such formal language is higher-order logic, a typed logic close to functional programming languages.  This is used as translation target of Haskabelle.

Both Haskabelle and Isabelle in combination allow to formally reason about Haskell programs, particularly verifying partial correctness.

The conversion employed by Haskabelle covers only a subset of Haskell, mainly since the higher-order logic of Isabelle has a more restrictive type system than Haskell.  A simple adaption mechanisms allows to tailor the conversion process to specific needs.

So far, Haskabelle is working, but there is little experience for its application in practice.  Suggestions and feedback welcome.

\FurtherReading
  \url{http://isabelle.in.tum.de/haskabelle.html}
  and \url{http://isabelle.in.tum.de/}
\end{hcarentry}

\end{document}
